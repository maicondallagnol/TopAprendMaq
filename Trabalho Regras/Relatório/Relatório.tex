\documentclass[
% -- opções da classe memoir --
12pt,				% tamanho da fonte
openright,			% capítulos começam em pág ímpar (insere página vazia caso preciso)
oneside,			% para impressão em recto e verso. Oposto a oneside
a4paper,			% tamanho do papel. 
% -- opções da classe abntex2 --
%chapter=TITLE,		% títulos de capítulos convertidos em letras maiúsculas
%section=TITLE,		% títulos de seções convertidos em letras maiúsculas
%subsection=TITLE,	% títulos de subseções convertidos em letras maiúsculas
%subsubsection=TITLE,% títulos de subsubseções convertidos em letras maiúsculas
% -- opções do pacote babel --
english,			% idioma adicional para hifenização
brazil				% o último idioma é o principal do documento
]{abntex2}

% ---
% Pacotes acrescentados
% ---
%\usepackage[portuguese, ruled, linesnumbered]{algorithm2e}
%\usepackage{algorithmic}

% ---
% Pacotes básicos 
% ---
\usepackage{lmodern}			% Usa a fonte Latin Modern			
\usepackage[T1]{fontenc}		% Selecao de codigos de fonte.
\usepackage[utf8]{inputenc}		% Codificacao do documento (conversão automática dos acentos)
\usepackage{lastpage}			% Usado pela Ficha catalográfica
\usepackage{indentfirst}		% Indenta o primeiro parágrafo de cada seção.
\usepackage{color}				% Controle das cores
\usepackage{url}				% Citar URLs
\usepackage{graphicx}			% Inclusão de gráficos
\usepackage{microtype} 			% para melhorias de justificação
\usepackage{booktabs}
\usepackage{multirow}
\usepackage[table]{xcolor}
\setlength{\aboverulesep}{0pt}
\setlength{\belowrulesep}{0pt}
\usepackage{scalefnt}
\usepackage{pdfpages}
% ---

% ---
% Pacotes adicionais, usados apenas no âmbito do Modelo Canônico do abnteX2
% ---
\usepackage{lipsum}				% para geração de dummy text
\usepackage{amssymb}			% para uso de símbolos matemáticos
% ---

% ---
% Pacotes de citações
% ---
\usepackage[brazilian,hyperpageref]{backref}	 % Paginas com as citações na bibl
\usepackage[alf]{abntex2cite}	% Citações padrão ABNT


% Pacotes adicionais Wal
\usepackage{subfig}  %para exibir figuras lado a lado
\usepackage{listings} % Para inclusao de codigos fontes
\usepackage{color}
\definecolor{mygreen}{rgb}{0,0.6,0}
\definecolor{mygray}{rgb}{0.5,0.5,0.5}
\definecolor{mymauve}{rgb}{0.58,0,0.82}
\lstset{ %
	backgroundcolor=\color{white},   % choose the background color; you must add \usepackage{color} or \usepackage{xcolor}
	basicstyle=\footnotesize,        % the size of the fonts that are used for the code
	breakatwhitespace=false,         % sets if automatic breaks should only happen at whitespace
	breaklines=true,                 % sets automatic line breaking
	captionpos=t,                    % sets the caption-position to bottom
	commentstyle=\color{mygreen},    % comment style
	deletekeywords={...},            % if you want to delete keywords from the given language
	escapeinside={\%*}{*)},          % if you want to add LaTeX within your code
	extendedchars=true,              % lets you use non-ASCII characters; for 8-bits encodings only, does not work with UTF-8
	frame=single,	                 % adds a frame around the code
	keepspaces=true,                 % keeps spaces in text, useful for keeping indentation of code (possibly needs columns=flexible)
	keywordstyle=\color{blue},       % keyword style
	language=Octave,                 % the language of the code
	otherkeywords={*,...},           % if you want to add more keywords to the set
	numbers=left,                    % where to put the line-numbers; possible values are (none, left, right)
	numbersep=5pt,                   % how far the line-numbers are from the code
	numberstyle=\tiny\color{mygray}, % the style that is used for the line-numbers
	rulecolor=\color{black},         % if not set, the frame-color may be changed on line-breaks within not-black text (e.g. comments (green here))
	showspaces=false,                % show spaces everywhere adding particular underscores; it overrides 'showstringspaces'
	showstringspaces=false,          % underline spaces within strings only
	showtabs=false,                  % show tabs within strings adding particular underscores
	stepnumber=1,                    % the step between two line-numbers. If it's 1, each line will be numbered
	stringstyle=\color{mymauve},     % string literal style
	tabsize=2,	                   	 % sets default tabsize to 2 spaces
	title=\lstname,                  % show the filename of files included with \lstinputlisting; also try caption instead of title
	numberbychapter=false
}


% --- 
% CONFIGURAÇÕES DE PACOTES
% --- 

\hyphenation{a-di-cio-nal-men-te}

% ---
% Configurações do pacote backref
% Usado sem a opção hyperpageref de backref
\renewcommand{\backrefpagesname}{Citado na(s) página(s):~}
% Texto padrão antes do número das páginas
\renewcommand{\backref}{}
% Define os textos da citação
\renewcommand*{\backrefalt}[4]{
	\ifcase #1 %
	Nenhuma citação no texto.%
	\or
	Citado na página #2.%
	\else
	Citado #1 vezes nas páginas #2.%
	\fi}%
% ---

%-----
%Adaptações do formato ABNT para UNESP
\addto\captionsbrazil{
	\renewcommand{\listfigurename}{Lista de Figuras}
	\renewcommand{\listtablename}{Lista de Tabelas}
	\renewcommand{\listadesiglasname}{Lista de Abreviaturas e Siglas}
	%\renewcommand{\folhadeaprovacaoname}{Folha de Aprovação}
	\renewcommand{\listadesimbolosname}{Lista de Símbolos}
}
%-----


% ---
% Informações de dados para CAPA e FOLHA DE ROSTO
% ---
\uppercase{\titulo{Trabalho de Classificação}}
\autor{Maicon Dall'Agnol}
\local{Rio Claro - SP}
\data{2019}
\orientador[Professora:]{Profa. Dra. Adriane Beatriz de Souza Serapião}
\instituicao{%
	UNIVERSIDADE ESTADUAL PAULISTA
	\par
	``J\'ULIO DE MESQUITA FILHO''
	\par
	Instituto de Geociências e Ciências Exatas - IGCE
	\par
	Curso de Bacharelado em Ciências da Computação}

\tipotrabalho{Trabalho de graduação}
% O preambulo deve conter o tipo do trabalho, o objetivo, 
% o nome da instituição e a área de concentração 
\preambulo{Trabalho de graduação da disciplica Tópico de Aprendizado de Máquina pelo Curso de Bacharelado em Ciências da Computação do Instituto de Geociências e Ciências Exatas da Universidade Estadual Paulista “Júlio de Mesquita Filho”, Câmpus de Rio Claro. }
% ---

% ---
% Configurações de aparência do PDF final

% alterando o aspecto da cor azul
\definecolor{blue}{RGB}{41,5,195}

% informações do PDF
\makeatletter
\hypersetup{
	%pagebackref=true,
	pdftitle={\@title}, 
	pdfauthor={\@author},
	pdfsubject={\imprimirpreambulo},
	pdfcreator={regras de associação},
	pdfkeywords={java}{proposta de trabalho}{unesp}, 
	colorlinks=false,       		% false: boxed links; true: colored links
	linkcolor=blue,          	% color of internal links
	citecolor=blue,        		% color of links to bibliography
	filecolor=magenta,      		% color of file links
	urlcolor=blue,
	bookmarksdepth=4
}
\makeatother
% --- 

% --- 
% Espaçamentos entre linhas e parágrafos 
% --- 

% O tamanho do parágrafo é dado por:
\setlength{\parindent}{1.3cm}

% Controle do espaçamento entre um parágrafo e outro:
\setlength{\parskip}{0.2cm}  % tente também \onelineskip

% ---
% compila o indice
% ---
\makeindex
% ---

% ----
% Início do documento
% ----
\begin{document}
	
	% Seleciona o idioma do documento (conforme pacotes do babel)
	%\selectlanguage{english}
	\selectlanguage{brazil}
	
	% Retira espaço extra obsoleto entre as frases.
	\frenchspacing 
	
	% ----------------------------------------------------------
	% ELEMENTOS PRÉ-TEXTUAIS
	% ----------------------------------------------------------
	
	\pretextual
	
 	  \begin{capa}
	\begin{center}
	\Large\imprimirinstituicao
	\end{center}

	\begin{center}
	\vspace*{3.4cm}
	\Large MAICON DALL'AGNOL
	\vspace*{1.5cm}
	
	\Large \textbf{\imprimirtitulo}
	
	\vspace*{3.4cm}
	
	\noindent Professora: Dra. Adriane Beatriz de Souza Serapião
	
	\vspace*{3.4cm}
	
	{\large\imprimirlocal}
	\par
	{\large\imprimirdata}
	\vspace*{1cm}
	\end{center}
  \end{capa}
	
	%\begin{folhaderosto}
	\begin{center}		
		
		\vspace*{\fill}\vspace*{\fill}
		\begin{center}
			\ABNTEXchapterfont\bfseries\Large\imprimirtitulo
		\end{center}
		\vspace*{\fill}
		
		\hspace{.45\textwidth}
		\begin{minipage}{.5\textwidth}
			\SingleSpacing
			\imprimirpreambulo
		\end{minipage}%		
		
		\vspace*{1cm}
		\begin{flushright}
			\noindent Aluno: Maicon Dall'Agnol
		\end{flushright}
		\vspace*{1cm}
		\begin{flushright}
			\noindent Professora: Dra. Adriane Beatriz de Souza Serapião
		\end{flushright}
		\vspace*{1cm}
				
				
		{\large\imprimirlocal}
		\par
		{\large\imprimirdata}
		\vspace*{1cm}
		
	\end{center}
\end{folhaderosto}
	
	%\include{errata}
	
	%\include{folhadeaprovacao}
	
	%\include{dedicatoria}
	
	%\include{agradecimentos}
	
	%\include{epigrafe}
	
	%\setlength{\absparsep}{18pt} % ajusta o espaçamento dos parágrafos do resumo
	%\include{resumo}
	
	%
%\pdfbookmark[0]{\listfigurename}{lof}
%\listoffigures*
%\cleardoublepage
% ---

% ---
% inserir lista de tabelas
% ---
%\pdfbookmark[0]{\listtablename}{lot}
%\listoftables*
%\cleardoublepage
% ---

% ---
% inserir lista de algoritmos
% ---
%\pdfbookmark[0]{Lista de Algoritmos}{loa}
%\listofalgorithms
%\cleardoublepage
% ---

% ---
% Lista de scripts
% ---
%\pdfbookmark[0]{Lista de Scripts}{loa}
%\lstlistoflistings
%\cleardoublepage

% ---
% inserir lista de abreviaturas e siglas
% ---
%\begin{siglas}
%	\item[KDD] Knowledge Discovery in Databases
%	\item[MD] Mineração da Dados
%	\item[MO] Medida Objetiva
%	\item[RA] Regra de Associação
%\end{siglas}
% ---

%\newpage
%\listofalgorithms*       % Lista de algoritmos
%\addcontentsline{toc}{section}{Lista de Algoritmos}

% ---
% inserir o sumario
% ---
\pdfbookmark[0]{\contentsname}{toc}
\tableofcontents*
\cleardoublepage
% ---

	
	% ----------------------------------------------------------
	% ELEMENTOS TEXTUAIS
	% ----------------------------------------------------------
	\textual
	
	%Capitulos
	\chapter{Introdução}\label{cap_intro}

Este trabalho consiste em aplicar o conhecimento de regras de associação adquirido na disciplina Tópicos: Aprendizado de Máquina, tendo assim como objetivo:

\begin{itemize}
\item Escolha dois datasets específicos para a tarefa de Regras de Associação.

\item Para cada dataset, você irá aplicar os algoritmos APriori e FP-Growth. Seu objetivo é minerar regras que contenham informações relevantes no dataset, seja porque alguma combinação de itens aparece com muita frequência, seja porque alguma combinação de itens não aparece com muita frequência mas está se destacando. Para isso, você deve variar os parâmetros de suporte, confiança e lift.

\item Existem duas métricas não estudadas em sala de aula, 'leverage' e 'conviction', que estão disponibilizadas no pacote mlxtend para regras de associação. Estude essas métricas e explique como elas podem contribuir para as análises dos seus datasets.

\item Compare os resultados gerados pelos dois algoritmos. Conclua sobre as diferenças encontradas nos resultados de cada dataset na aplicação dos dois algoritmos.
\end{itemize}

	\chapter{Desenvolvimento}\label{cap_desenv}

Para o desenvolvimento das atividades inicialmente foram escolhidos duas bases bases de dados. A primeira base a ser utilizada corresponde a dados de \textit{reviews} de um E-Commerce de Roupas Femininas contendo informações como idade, avaliação, categoria que o produto pertence, entre outras; A segunda base é composta dados educacionais que é coletado do sistema de gerenciamento de aprendizado, nela há dados como genero, nacionalidade, número de vezes que o aluno levanta a mão, entre outros dados.

\section{Pré-processamento e Visualização}
Ambas bases haviam dados categóricos e numéricos, na primeira base alguns atributos numéricos foram removidos e outros foram discretizados, ainda na primeira base também foram removidas transações que continham item (atributos=valor) faltantes; Para a segunda base visualizou através de um \textit{boxplot} que os dados numéricos estavam variando de 0 a 100 com uma média diferente para cada atributo, desta forma todos os dados numéricos foram discretizados em categorias binárias (abaixo ou acima média do atributo).

Para a aplicação dos algoritmos também transformou-se os dados do formato dataframe para um array da biblioteca Numpy.

\section{Regras de Associação}

Para aplicação do algoritmo Apriori inicialmente transformou-se os dados para um formato transacional utilizando TransactionEncoder, dessa forma utilizou-se o algoritmo apriori implementado na biblioteca mlxtend para extração dos \textit{itemsets} frequentes e $association_rules$ para extração das regras com confiança acima do corte.

Para aplicação Fp-Growth primeiramente implementou-se duas funções cujo propósito era: a. verificação de suporte e b. aplicação, cálculos das medidas. Nesta etapa algumas dificuldades foram encontradas uma vez que a implementação do algoritmo $fp_growth$ baixada pelo indexador de pacotes de python PyPI não retornava todas as medidas, portanta algumas tiveram que ser calculadas o que gerou alguns problemas ao se buscar os suportes do antecedente e consequente.

\section{Avaliação}

	%\chapter{Conclusão}\label{cap_conclu}

Aqui vai a conclusão
	
	
	% ----------------------------------------------------------
	% Finaliza a parte no bookmark do PDF
	% para que se inicie o bookmark na raiz
	% e adiciona espaço de parte no Sumário
	% ----------------------------------------------------------
	\phantompart
	
	%\chapter{Conclusão}
	
	
	% ----------------------------------------------------------
	% ELEMENTOS PÓS-TEXTUAIS
	% ----------------------------------------------------------
	\postextual
	% ----------------------------------------------------------
	
	% ----------------------------------------------------------
	% Referências bibliográficas
	% ----------------------------------------------------------
	\includepdf[page=-]{Regras}
	\includepdf[page=-]{Regras2}
	
	%\bibliography{references}
\end{document}
