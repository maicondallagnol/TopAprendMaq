\chapter{Introdução}\label{cap_intro}

Este trabalho consiste em aplicar o conhecimento de regras de associação adquirido na disciplina Tópicos: Aprendizado de Máquina, tendo assim como objetivo:

\begin{itemize}
\item Escolha dois datasets específicos para a tarefa de Regras de Associação.

\item Para cada dataset, você irá aplicar os algoritmos APriori e FP-Growth. Seu objetivo é minerar regras que contenham informações relevantes no dataset, seja porque alguma combinação de itens aparece com muita frequência, seja porque alguma combinação de itens não aparece com muita frequência mas está se destacando. Para isso, você deve variar os parâmetros de suporte, confiança e lift.

\item Existem duas métricas não estudadas em sala de aula, 'leverage' e 'conviction', que estão disponibilizadas no pacote mlxtend para regras de associação. Estude essas métricas e explique como elas podem contribuir para as análises dos seus datasets.

\item Compare os resultados gerados pelos dois algoritmos. Conclua sobre as diferenças encontradas nos resultados de cada dataset na aplicação dos dois algoritmos.
\end{itemize}
