\chapter{Proposta}

A evasão escolar é um problema que afeta todas as etapas escolares, sendo agravada no ensino médio. Este trabalho tem como proposta a utilização de Extended Association Rule Network (ExARN) para exploração de uma base de dados escolares de um período de 5 anos.

A \textit{Association Rule Network} (ARN), tem como objetivo reduzir o número de regras a serem exploradas pelo usuário, tendo assim um item específico que terá um Directed Acyclic Graph (DAG) construído utilizando todas as regras que se relacionam com ele direta ou indiretamente.

O ExARN~\citeonline{DBLP:conf/bracis/PaduaCCR18} se propõe como um método extendido das Association Rule Network (ARN), onde o item-objeto não limita-se a um único item, podendo ter diversos itens.

Como mostrado em \cite{DBLP:conf/ismir/PaduaCRS18}, as ExARN podem ser uteis como método de exploração mais ampla de hipóteses, deste modo, este trabalho se propõe a utilizar o método para exploração da evasão escolar.