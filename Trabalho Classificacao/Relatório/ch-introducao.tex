\chapter{Introdução}\label{cap_intro}

Este trabalho consiste em aplicar o conhecimento de Classificadores adquirido na disciplina Tópicos: Aprendizado de Máquina. Tem-se como objetivo:

\begin{itemize}
\item Escolha dois datasets rotulados.
\item Realize a análise estatística, visualização e pré-processamento dos dados.
\item Realize os experimentos criando duas bases de teste distintas:
\begin{itemize}
\item considerando todos os atributos do dataset;
\item selecionando alguns atributos e descartando outros.
\end{itemize}
\item Aplique três métodos de classificação distintos nas duas bases acima referentes a cada dataset.
\item Para cada dataset, em cada uma das bases, analise os resultados segundo medidas de qualidade de classificação, usando índices de validação externa (acurácia, recall, precisão, F-measure, índice Kappa) e cruva ROC.
\item Proponha uma maneira adicional de comparar os resultados obtidos além das medidas acima.
\item Compare e interprete os resultados dos dois experimentos em cada dataset.
\item Faça tabela com as medidas de validação	
\end{itemize}
