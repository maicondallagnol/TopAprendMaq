\chapter{Introdução}\label{cap_intro}

Este trabalho consiste em aplicar o conhecimento em regressão (predição) adquirido na disciplina Tópicos: Aprendizado de Máquina, tendo assim como objetivo:

\begin{itemize}
\item Escolher dois conjuntos de dados para trabalhar o problema de regressão. Separe cada \textit{dataset} em conjunto de treinamento e conjunto de teste. Explique o seu critério de separação e o método utilizado.
\item Você deverá implementar soluções para cada \textit{dataset} usando:
\begin{itemize}
	\item regressão linear (ou regressão múltipla)
	\item regressão polinomial
	\item SVR (kernels linear, sigmoide, RBF e polinomial)
	\item rede neural (MLP ou RBF). 
\end{itemize}

\item Descrever os parâmetros/arquiteturas de cada modelo.
\item Compare os resultados (para treinamento e teste) com as medidas de desempenho SEQ, EQM, REQM, EAM e $r^2$, e verifique qual a melhor opção dentre os métodos implementados que melhor se ajusta a seus dados.
\item Você deverá fazer a visualização dos dados originais com os dados ajustados em cada experimento, tanto para o conjunto de treinamento quanto para o de teste. Os gráficos devem conter títulos nos eixos e legenda. Comente os resultados encontrados na visualização.
\end{itemize}
